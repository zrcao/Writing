\documentclass[letterpaper,11pt]{article}
%\usepackage[textwidth=6.5in,textheight=8.5in]{geometry}
\usepackage[margin=1.0in]{geometry}
\setlength{\headheight}{28pt}
\usepackage{graphicx}
\usepackage{fancyhdr}
\usepackage{array}
\usepackage{amsmath}
%document font (Times)
\usepackage{tgtermes}
\usepackage[T1]{fontenc}
%\usepackage{mathptmx}
\usepackage{enumitem}
\usepackage{tabularx}
\usepackage[usenames,dvipsnames,svgnames,table]{xcolor}
\usepackage{colortbl,hhline}
\definecolor{c1}{rgb}{0.30980, 0.50588, 0.73725}
\definecolor{c2}{rgb}{0.82353, 0.87843, 0.92941}
\usepackage{longtable}
\usepackage{dsfont}
% Save space between sections
\usepackage[compact]{titlesec}
\titlespacing{\section}{0pt}{*1}{*1}
\titlespacing{\subsection}{0pt}{*1}{*1}
\titlespacing{\subsubsection}{0pt}{*1}{*1}
% For including PDF pages
\usepackage{pdfpages}
\usepackage{yfonts}

%\setlength{\topskip}{20pt}
%\hoffset=-0.74in
%\voffset= -0.8in
%\textwidth=6.5in
%\textheight=8.5in
%\marginparwidth=0pt

%\fancyhead{}
%\fancyfoot{}
%\pagenumbering{Roman}
\lhead{}
%\lhead{\includegraphics[width=0.25\textwidth]{../figures/C-3logo.png}}
\chead{}
%\chead{Proposal Number\\A151-020-0273}
\rhead{}
%\rhead{Topic Number\\A15-020}
%\cfoot{\textcopyright~C-3 Comm Systems LLC. Proprietary.}
\cfoot{}
\rfoot{\thepage}

\pagestyle{fancy}

\title{\bf{From Maxwell To Channel State Information}}
\author{Zhongren Cao, Ph.D.}

\let\bs\boldsymbol

\begin{document}

\thispagestyle{fancy}

\maketitle

\section{The Derivation of Wireless Channel Representation}

Let $x(t)$ denote the transmitted signals and $y(t)$ denote the received signals. 

\subsection{Free Space Line of Sight (LOS)}

First, let's consider a fixed antenna radiating a single sinusoid signal, $x(t)=cos2\pi ft$, into the free space. Since the electric field and magnetic field are proportional to each other, it is sufficient to know only one of them. Assuming that the location of the transmitting antenna is at the original point. For a far-field point $\bf{u}=(\theta, \psi, r)$ in the space, we have the electric field value as
\begin{equation}
E(f,t,{\bf{u}})=\frac{\alpha_s(\theta, \psi, f)\cos2\pi f(t-r/c)}{r}.
\end{equation}
Here, $r$ is the distance between the point $\bf{u}$ and the transmit antenna, $c$ is the speed of light, and $\alpha_s(\theta, \psi, f)$ is the radiation pattern of the transmit antenna at frequency $f$ and the direction $(\theta, \psi)$.

Next, let's assume we have a receive antenna at point $\bf{u}$, the signal received by the antenna can be represented as
\begin{equation}\label{rxsignal}
y(t)=\frac{\alpha(\theta,\psi,f)\cos2\pi f(t-r/c)}{r}.
\end{equation}
where $\alpha(\theta,\psi,f)=\alpha_s(\theta, \psi, f)\alpha_r(\tilde{\theta}, \tilde{\psi}, f)$ the product of the transmit antenna pattern and the receive antenna pattern at frequency $f$. The direction $\tilde{\theta}, \tilde{\psi}$ is the incoming direction of the signal from the receive antenna point of view. \\

\fbox{\parbox{0.9\textwidth}{{\textbf{Assumption I}}: Placing a receive antenna at point $\bf{u}$ inevitably changes the electric field in the vicinity of $\bf{u}$. Here, we made an assumption that the change of the electric field at point $\bf{u}$ is taken into account by the pattern of the receive antenna at $\bf{u}$.}}

~\\

Let $a=\alpha(\theta, \psi, f)/r$ and $\hat{\tau}=r/c$. $a$ is the signal attenuation and $\hat{\tau}$ is the delay. The received signal in Eq.~(\ref{rxsignal}) can be re-written as
\begin{equation}\label{rxsignal_tau}
y(t)=a\cos2\pi f(t-\hat{\tau})=ax(t-\hat{\tau}).
\end{equation}
Let's treat $x(t)$ as the system input and $y(t)$ as the system output. Eq.~(\ref{rxsignal_tau}) indicates that the system response at the receive antenna is linear. This is a crucial conclusion for wireless channel modeling. \\

\fbox{\parbox{0.95\textwidth}{{\textbf{Conclusion I}}: Wireless channel is a linear system. Therefore, the received field at $\bf{u}$ in responses to a weighted sum of transmitted waveform is the weighted sum of responses to individual waveforms.}}


\subsection{Received Sinusoid in Scattering and Mobile Environment}

In an scattering environment, there may or may not be a LOS path from the transmitter to the receiver. The received signal is the sum of multiple reflected rays from scattering objects in the surround environment and, if it exists, the LOS ray. Let's focus on the $i$th ray here, which is a non-LOS (NLOS) ray. Due to the mobility, both the attenuation and the delay are functions of time $t$. Based on Eq.~(\ref{rxsignal_tau}), the receive signal from the $i$th ray is
\begin{equation}\label{rxsignal_oneray}
\displaystyle y_i(t)=a_i(f,t)\cos\big(2\pi f[t-\hat{\tau}_i(t)]-\phi_i\big),
\end{equation}
where $\phi_i$ accounts for the overall phase change at the transmitter, reflectors and the receiver. Let
$$
\tau_i(f, t)=\hat{\tau_i(t)}+\frac{\phi_i}{2\pi f},
$$ 
we can represent Eq.~(\ref{rxsignal_oneray}) in a similar format as Eq.~(\ref{rxsignal_tau}) in the following.
\begin{equation}
y_i(t)=a_i(f,t)\cos2\pi f\big(t-\tau_i(f, t)\big).
\end{equation}
Note that $\tau_i$ is a function of both frequency $f$ and time $t$. It can be view as the effective propagation delay of the $i$th ray.

The received signal is the sum of all rays, i.e.,
\begin{equation}
y(t)=\displaystyle\sum_i a_i(f,t)\cos2\pi f\big(t-\tau_i(f, t)\big).
\end{equation}


%Let $h_i(\tau;t)=a_i(f,t)\sigma\big(t-\tau_i(f,t)\big)$

\subsection{Band-limited Signals}

In the previous section, we focused on a single sinusoid input. The communication signal $x(t)$ always occupies a limited bandwidth $W=f_H-f_L$. The center carrier frequency is $f_c=(f_L+f_H)/2$. The signal x(t) sent from the transmit antenna can be represented by 
\begin{equation}
\displaystyle x(t)=\int_{f_L}^{f_H}X(f)e^{j2\pi ft}df+\int_{-f_H}^{-f_L}X(f)e^{j2\pi ft}df,
\end{equation}
where $X(f)$ is the Fourier Transform of $x(t)$. Since $x(t)$ is a real signal, $X(f)$ satisfies $X(f)=X^{*}(-f)$. Therefore, we have
\begin{equation}
\begin{array}{rl}
\displaystyle x(t)&=\displaystyle \int_{f_L}^{f_H}\left(X(f)e^{j2\pi ft}+X^{*}(f)e^{-j2\pi ft}\right)df\\
\displaystyle &=\displaystyle 2\int_{f_L}^{f_H}{\textfrak{Re}}\left[X(f)e^{j2\pi ft}\right]df\\
&=\displaystyle 2\int_{f_L}^{f_H}\bigg({\textfrak{Re}}\left[X(f)\right]\cos2\pi ft-{\textfrak{Im}}\left[X(f)\right]\sin2\pi ft\bigg)df\\
\end{array}
\end{equation}
Therefore, $x(t)$ is continuous integration of sinusoids between frequency $[f_L, f_H]$. According to \textbf{Conclusion I} above, the receive signal $y(t)$ in response to an arbitrary band limited signal $x(t)$ is 
\begin{equation}\label{frq_dependent}
%y(t)=\displaystyle\sum_i a_i(f,t)x\big(t-\tau_i(f, t)\big).
y(t)=\displaystyle2\int_{f_L}^{f_H}\sum_ia_i(f,t)\bigg({\textfrak{Re}}\left[X(f)\right]\cos2\pi f\big(t-\tau_i(f,t)\big)-{\textfrak{Im}}\left[X(f)\right]\sin2\pi f\big(t-\tau_i(f,t)\big)\bigg)df
\end{equation}

The attenuations $a_i(f,t)$ and the effective propagation delay $\tau_i(f,t)$ are slow varying along the frequency axis. Let's assume  that the bandwidth of signal is much smaller than the center carrier frequency, i.e. $W\ll f_C$. In other words, signal $x(t)$ is a narrow band signal. Now, we are going to the second important assumption. \\

\fbox{\parbox{0.9\textwidth}{{\textbf{Assumption II}}: For narrow band signals, at any given time instance $t$, the attenuators $a_i(f,t)$ and the effective propagation delay $\tau_i(f,t)$ are constant over the signal bandwidth $W$.}}

~~\\

Based on {\textbf{Assumption II}}, the model in Eq.~(\ref{frq_dependent}) is simplified to
\begin{equation}\label{frq_independent}
\begin{array}{rl}
y(t)&=\displaystyle2\sum_ia_i(t)\int_{f_L}^{f_H}\bigg({\textfrak{Re}}\left[X(f)\right]\cos2\pi f\big(t-\tau_i(t)\big)-{\textfrak{Im}}\left[X(f)\right]\sin2\pi f\big(t-\tau_i(t)\big)\bigg)df\\
&=\displaystyle\sum_i a_i(t)x\big(t-\tau_i(t)\big).\\
\end{array}
\end{equation}
The wireless channel is defined as
\begin{equation}\label{passbandChannelDef}
\boxed{h(\tau;t)=\displaystyle\sum_ia_i(t)\delta\bigg(\tau-\tau_i(t)\bigg),}
\end{equation}
where $\delta(\tau)$ is the Dirac delta function on the $\tau$ axis. Using the wireless channel definition in Eq.~{(\ref{passbandChannelDef}}), Eq.~(\ref{frq_independent}) can be expressed as linear system response between input $x(t)$ and output $y(t)$ as
\begin{equation}
y(t)=\displaystyle\int_{-\infty}^{\infty}h(\tau;t)x(t-\tau)d\tau.
\end{equation}
The above two equations are very neat representations. It says that the effect of mobile users, arbitrarily moving reflectors and absorbers, and all the complexities of solving Maxwell's equations, finally reduce to an input/output relation represented as the impulse response of a linear time-varying filter. This is the essence of wireless channel.

The Doppler shift is not immediately evident in the representation of Eq.~(\ref{passbandChannelDef}). In fact, for the $i$th path, the effective delay is a function of time $t$. Its derivative against $t$ is $\tau_i'(t)$, and the Doppler shift on the $i$th path for frequency $f$ is $-f\tau'_i(t)$. 

For the special case when the environment is static so the attenuators $a_i(t)$ and the effective propagation delay $\tau_i(t)$ do not vary over time, we have the linear time-invariant channel model with the impulse response representation as
\begin{equation}
h(\tau)=\displaystyle\sum_ia_i\delta(\tau-\tau_i).
\end{equation}



\subsection{Baseband Equivalent Channel Model}

\subsection{Discrete-time Baseband Channel Model}

\section{Channel State Information Generation from SciX3 Software}

\bibliographystyle{IEEEtran}
\bibliography{../BibTex/ZCaoBibCollection.bib}

\end{document} 
