\documentclass[letterpaper,11pt,onecolumn]{article}
\usepackage{layout}
\usepackage{graphicx}
\usepackage{fancyhdr}
\usepackage{array}
\usepackage{amsmath}
%document font (Times)
\usepackage{tgtermes}
\usepackage[T1]{fontenc}
%\usepackage{mathptmx}


\hoffset=-72.27pt
\voffset=-72.27pt
\textwidth=506.30pt
\textheight=650pt
\marginparwidth=0pt

%\fancyhead{}
%\fancyfoot{}
\pagenumbering{Roman}
\lhead{\includegraphics[width=0.25\textwidth]{../figures/C-3logo.png}}
\chead{}
\rhead{}
%\cfoot{Mailing Address: \\Tel: 1(703)}


\pagestyle{fancy}

\begin{document}

%TT\layout{}

\thispagestyle{fancy}

\begin{center}
~\\
~\\
~\\
{\Large{\bf{Large-Scale Android Network Experiment Harness\\
~~~~~~ --- Use Cases and Requirements}}}\\
~\\
\vspace{0.1in}
Zhongren Cao\\
zcao@c3commsystems.com
\end{center}

\vspace{0.1in}

\section{IDENTIFICATION \& SIGNIFICANCE OF THE OPPORTUNITY}

% Talk about the importance of tactical edge networks

Improving the networking capacity of tactical edge mobile networks has a direct impact on mission successes. Wireless networks are interference limited due to the fact that multiple users need to share the transmission resources. Traditionally, wireless networks share resources among multiple users by dividing frequency and/or time into orthogonal units, such as in TDMA, FDMA and OFDMA. Each resource unit is only used by a single transmission every time in order to avoid interference among multiple transmissions. In addition, frequency reuse is widely adopted in commercial cellular systems and military systems to increase the network system capacity. In frequency reuse, the same frequency/time resource unit is shared by multiple transceiver pairs that are far apart geographically such that mutual interference is not significant due to path loss. Vector-based signaling are used in code division multiple access (CDMA) using orthogonal codes and spatial division multiple access (SDMA) with multiple antenna arrays. Vector space provides an additional dimension of transmission resource. Therefore, the same time/frequency resource unit could be shared among multiple users as long as they are orthogonal to each other in the vector space. 

All aforementioned approaches are built upon a basic principle that, in order to avoid interference, closely located nodes in a wireless network can only use orthogonal resource units for simultaneous transmissions. The capacity of a wireless network is thus limited by the number of available orthogonal resource units. 

Interference alignment is an emerging approach, which breaks away from the above fundamental principle. In interference alignment, the signaling of multiple closely located transceiver pairs are jointly designed, such that at each individual receiver all interference aligned into a subspace that is orthogonal to the desired signal. The number of interferences is more than the rank of the aligned subspace. This is explained graphically in Fig.*. ... Using interference alignment, from the network point of view, the number of simultaneous transmissions within the interference range can be more than the number of available orthogonal resource units. Thus, the overall network throughput is increased.


at least in theory, shows it can significantly improve the network capacity. The theoretic concept of interference has been extensively explained in many academia publications. The theoretic framework for IA has several idealistic assumptions that need to be addressed before apply IA in practical systems. First, since IA jointly design the transmission signals for multiple transmitters, the channel state information (CSI) is central to calculating IA beamforming vectors. As a result, IA incurs significant overhead for CSI estimations and feedback. In mobile fast-fading environment, the overhead of CSI acquisition can limit or cancel out the gains of IA. 


However, how to materialize the capacity gain promised by IA in an operational network is still a challenging problem. In particular, tactical mobile ad hoc network (TMACN) poses several challenges to apply IA. 

Several challenges have to be addressed in order to reap the benefits of IA in operational tactical mobile ad hoc network. 

First, tactical networks are inherently heterogeneous. Dismounted soldier nodes and vehicle mounted radio transceivers have different requirements in terms of size, weight and power (SWaP). Hence, different nodes are equipped with different number of antennas. For example, the AN/PRC-154 Rifleman Radio that is being fielded now is a single transceiver radio. The AN/PRC-155 Manpack radio has two radio channels. More RF channels are potentially  available on vehicle mounted radios. 

Second, unlikely commercial systems, TMACN usually doesn't have a centralized control plane. Therefore, IA relies on distributed coordination among peer nodes. 

Third,  technology evolution instead of revolution is better for TMACN.

Therefore, a distributed 

First, tactical network is inherently heterogeneous.  size weight and power (SWaP) requirements 

% Situational awareness at tactical edge requires  higher data throughput

% Interference alignment overview and its potential application to improve the throughput of tactical edge network

% General challenges of applying interference alignment 

% Challenges posed by the tactical edge networks
%  1. Hybrid nodes (vehicle node vs. dismounted nodes) from one antenna to multiple antennas
%  2. Hierarchical network architecture

It requires several steps to incorporate IA into the TMACN operations. The first step is to identify IA opportunities. 

In this project, we propose to design a software suite, which enables distributed interference alignment cooperations among tactical edge mobile nodes.  Since most existing military waveforms operate in slotted or TDMA mode, THE SYSTEM assumes a operational slotted network without IA as the baseline. In order to be backward compatible with existing tactical protocols,  where initial slot assignment is assumed to be performed by methods used in existing TMACN, such as SRW. Thus, each node possesses a series of time slots. The system level gain from IA comes from enabling other transceiver pairs to share time slots assigned to node ??. 

In THE SYSTEM, IA cooperations among multiple nodes are accomplished in four steps. 

First, IA opportunity identification. Each node maintains a neighbor table based on 

Second, IA mode selection.

Third, training and feedback based on the mode selected.

Fourth, 


\section{PHASE I TECHICAL OBJECTIVES}

\section{PHASE I WORK PLAN}

\section{RELATED WORK}

\section{RELATIONSHIP WITH FUTURE RESEARCH OR RESEARCH AND DEVELOPMENT}

\section{COMMERCIALIZATION STRATEGY}

\section{KEY PERSONNEL}

\section{FACILITIES/EQUIPMENT}

\section{CONSULTANTS}

\section{PRIOR, CURRENT OR PENDING SUPPORT}

\end{document} 
