\documentclass[letterpaper,11pt,onecolumn]{article}
\usepackage{layout}
\usepackage{graphicx}
\usepackage{fancyhdr}
\usepackage{array}
\usepackage{amsmath}
%document font (Times)
\usepackage{tgtermes}
\usepackage[T1]{fontenc}
%\usepackage{mathptmx}


\hoffset=-72.27pt
\voffset=-72.27pt
\textwidth=506.30pt
\textheight=650pt
\marginparwidth=0pt

%\fancyhead{}
%\fancyfoot{}
\pagenumbering{Roman}
\lhead{\includegraphics[width=0.25\textwidth]{../figures/C-3logo.png}}
\chead{}
\rhead{}
%\cfoot{Mailing Address: \\Tel: 1(703)}


\pagestyle{fancy}

\begin{document}

%TT\layout{}

\thispagestyle{fancy}

\begin{center}
~\\
~\\
~\\
{\Large{\bf{Large-Scale Android Network Experiment Harness\\
~~~~~~ --- Use Cases and Requirements}}}\\
~\\
\vspace{0.1in}
Zhongren Cao\\
zcao@c3commsystems.com
\end{center}

\vspace{0.1in}

\section{IDENTIFICATION \& SIGNIFICANCE OF THE OPPORTUNITY}

% Talk about the importance of tactical edge networks

Improving the networking capacity of tactical edge mobile networks has a direct impact on mission successes. Traditionally, wireless network shares its resources among multiple users by dividing transmission resource into orthogonal units, such as TDMA and FDMA. In addition, spatial reuse 

Interference alignment is an emerging approach that, at least in theory, shows it can significantly improve the network capacity. The theoretic concept of interference has been extensively explained in many academia publications. However, how to materialize the capacity gain promised by IA in an operational network is still a challenging problem. In particular, tactical mobile ad hoc network (TMACN) poses several challenges to apply IA. 

First, tactical networks are inherently heterogeneous. Dismounted soldier nodes and vehicle mounted radio transceivers have different requirements in terms of size, weight and power (SWaP). Hence, different nodes are equipped with different number of antennas. For example, the AN/PRC-154 Rifleman Radio that is being fielded now is a single transceiver radio. The AN/PRC-155 Manpack radio has two radio channels. More RF channels are potentially  available on vehicle mounted radios. 

Second, unlikely commercial systems, TMACN usually doesn't have a centralized control plane. Therefore, IA relies on distributed coordination among peer nodes. 

Third,  technology evolution instead of revolution is better for TMACN.

Therefore, a distributed 

First, tactical network is inherently heterogeneous.  size weight and power (SWaP) requirements 

% Situational awareness at tactical edge requires  higher data throughput

% Interference alignment overview and its potential application to improve the throughput of tactical edge network

% General challenges of applying interference alignment 

% Challenges posed by the tactical edge networks
%  1. Hybrid nodes (vehicle node vs. dismounted nodes) from one antenna to multiple antennas
%  2. Hierarchical network architecture

It requires several steps to incorporate IA into the TMACN operations. The first step is to identify IA opportunities. 

In this project, we propose to design a software suite, which enables distributed interference alignment cooperations among tactical edge mobile nodes.  Since most existing military waveforms operate in slotted or TDMA mode, THE SYSTEM assumes a operational slotted network without IA as the baseline. In order to be backward compatible with existing tactical protocols,  where initial slot assignment is assumed to be performed by methods used in existing TMACN, such as SRW. Thus, each node possesses a series of time slots. The system level gain from IA comes from enabling other transceiver pairs to share time slots assigned to node ??. 

In THE SYSTEM, IA cooperations among multiple nodes are accomplished in four steps. 

First, IA opportunity identification. Each node maintains a neighbor table based on 

Second, IA mode selection.

Third, training and feedback based on the mode selected.

Fourth, 


\section{PHASE I TECHICAL OBJECTIVES}

\section{PHASE I WORK PLAN}

\section{RELATED WORK}

\section{RELATIONSHIP WITH FUTURE RESEARCH OR RESEARCH AND DEVELOPMENT}

\section{COMMERCIALIZATION STRATEGY}

\section{KEY PERSONNEL}

\section{FACILITIES/EQUIPMENT}

\section{CONSULTANTS}

\section{PRIOR, CURRENT OR PENDING SUPPORT}

\end{document} 
